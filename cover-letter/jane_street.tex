\documentclass{article}

%%%% Begin package imports %%%%%%%%%%%%%%%%%%%%%%%%%%%%%%%%%%%%%%%%%%%%%%%%%%%%

% Language and font encodings
\usepackage[english]{babel}
% Set page size and margins
\usepackage[a4paper,top=3cm,bottom=2cm,left=3cm,right=3cm,marginparwidth=1.75cm]{geometry}

\setcounter{secnumdepth}{0}

%%%%%%%%%%%%%%%%%%%%%%%%%%%%%%%%%%%%%%%%%%%%%%%%%%%%%% End package imports %%%%
%%%% Begin document contents %%%%%%%%%%%%%%%%%%%%%%%%%%%%%%%%%%%%%%%%%%%%%%%%%%


\begin{document}

\section{Cover Letter}

Each project I have worked on has taught me lessons. I believe that the traits
which I value most in myself are self-reflection, communication, and
management, and I believe that these skills make me stand out most as a
professional, above any specific technical skills. I would like to justify why
I value each of these traits, and give some anecdotes of my experience in doing
so.\\

Self-reflection has brought about so much of my personal growth.
I use self-reflection in the form of writing up regular reviews and learning
from each of my mistakes and experiences, good and bad, and spotting trends in
my performance. This was something I picked up and honed as a development
intern at JPMorgan Chase, and this \textit{learning how to learn} was perhaps
my biggest lesson from my time there. As a result of weekly reflection,
planning, and record keeping, I was able to take on a large amount of
responsibility and complexity -- organizing and chairing meetings with clients,
my team, and supervisors, and ultimately delivering a batch to production.
Having self-awareness of my own strengths and weaknesses, and how to improve on
these, has remained invaluable to me. Moreover, I am certain that
self-reflection will allow me to play to my strengths, learn from my mistakes,
and make sustained growth throughout my future employment.\\

In every project I have been on, the project's success can be attributed in
great part to the group's successful communication. I have learned the value of
communicating effectively and concisely, in particular from my time as
president of my team in the International Space Settlement Design Competition
(ISSDC). Presiding over a team of 50, I saw that setting clear objectives for
each team, effective organizational structure, and exchanging ideas were the
decisive factors of the competition, and that every group's final products were
a product of these factors. Through this I have come to greatly value open and
clear communication with my coworkers and clients.\\

How to manage teams, and myself, has been a key factor in all the projects I
have worked on. I feel that understanding the role of each person and team
inside an organization will make teams efficient and cooperative when these are
used to structure the project and organization. On top of using people for
their strengths, I believe aligning incentives personal and group incentives is
decisive for projects. I found that my team within JPMorgan showed what the
right management and incentives can do to maximize team cohesion and effiency.
I admired how thoroughly my team had adopted agile techniques, the
short-and-sweet approach to meetings, and a desire to shine as individuals and
in our team's final product, which contrasted to many of my less efficient
experiences in teams. Above all, the project leader's experience and direction
felt non-egotistical, well-informed, and thoughtful, as I felt was best.\\

My philosophy of software engineering has grown to be treating code with
certain values in mind. To me, good software works, with assurance that it does
what it is expected to, is clear, and is flexible. Meanwhile, I place a great
personal value on high performance code, which behaves exactly as expected, and
is well optimized for the environment it will run in. These values lead me to
believe that I would be well suited to be a <job\_name> at Jane Street.\\

Overall, I feel that the <company\_culture> and <job\_comment> mean that I will
thrive in this position. I feel that the role plays to my strengths in self-
reflection, communication, and management, as well as to my strengths as a
software developer.\\

I look forward to hearing from you in due course.\\

Kind regards,\\

Andrew J. Young\\

\end{document}

%%%%%%%%%%%%%%%%%%%%%%%%%%%%%%%%%%%%%%%%%%%%%%%%%%%% End document contents %%%%
